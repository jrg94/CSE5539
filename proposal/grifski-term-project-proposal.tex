\documentclass[12pt, a4paper]{article}
\setlength{\oddsidemargin}{0.5cm}
\setlength{\evensidemargin}{0.5cm}
\setlength{\topmargin}{-1.6cm}
\setlength{\leftmargin}{0.5cm}
\setlength{\rightmargin}{0.5cm}
\setlength{\textheight}{24.00cm}
\setlength{\textwidth}{15.00cm}
\parindent 0pt
\parskip 5pt
\pagestyle{plain}

\title{Project Proposal}
\author{}
\date{}

\newcommand{\namelistlabel}[1]{\mbox{#1}\hfil}
\newenvironment{namelist}[1]{%1
\begin{list}{}
    {
        \let\makelabel\namelistlabel
        \settowidth{\labelwidth}{#1}
        \setlength{\leftmargin}{1.1\labelwidth}
    }
  }{%1
\end{list}}

\begin{document}
\maketitle

\begin{namelist}{xxxxxxxxxxxx}
\item[{\bf Title:}]
	Music Data Mining for Data Visualization
\item[{\bf Author:}]
	Jeremy Grifski
\item[{\bf Instructor:}]
	Professor DeLiang Wang
\end{namelist}

\section*{Background}

As a first-year PhD student, my research background is very limited. That said,
I am interested in doing research in the area between music, education, gaming,
and data visualization.

To supplement these interests, I took three game development courses, a
computer graphics course, and a modeling and simulation course during undergrad.
In addition, I recently took a real-time rendering course, and I am
currently taking a data visualization course and a graphics seminar.

In terms of technical experience, I spent two years in industry working for
General Electric Transportation as a part of the Edison Engineering Development
Program. In the span of two years, I managed to rotate through various roles
including software engineer and prognostics engineer. The first role allowed
me to work on camera systems while the second role gave me experience with
some basic data analytics.

\section*{Aim}

The aim of this project is to explore different audio signal processing methods
as a mode of data mining for the purposes of visualization. For example, I am
interested in collecting data such as loudness, onset density, and auditory
roughness~\cite{jeong}. In addition, I'd like to explore pitch~\cite{cuadra}\cite{rabiner},
key~\cite{zhu}\cite{chai}, and onset detection~\cite{bello}.

In addition, with all the data, I'd like to do a bit of visualization to see
if there are any interesting trends between music categories. For example,
is rock music audio generally more rough than pop music? Do some artists
leverage key changes more than others? How has loudness varied over the
decades?

\section*{Method}

Over the course of the semester, there are several small tasks that I would like
to complete. The list below details a set of features to be implemented in the
final version of the music data mining and visualization tool:

\begin{itemize}
  \item Music directory selection
  \item Recursive music directory traversal
  \item Music file data modeling using Python classes
  \item Music file metadata mapping to data model
  \begin{itemize}
    \item Length
    \item Genre
    \item Artist
    \item Year
    \item Bitrate
  \end{itemize}
  \item Mine music file for signal data
  \begin{itemize}
    \item Loudness
    \item Auditory Roughness
    \item Onset Density
    \item Pitch
    \item Key
  \end{itemize}
  \item Aggregate music data by category (see metadata)
  \item Visualize filtered music data
  \begin{itemize}
    \item Average auditory roughness vs. genre
    \item Average key change count vs. artist
    \item Average loudness vs. time
  \end{itemize}
\end{itemize}

Overall, the plan is to build a functioning audio data mining tool which can be
leveraged for data visualization.

\section*{Software and Hardware Requirements}

For this project, I intend to use Python for both signal processing and
visualization. Signal processing can be done using the scipy library, and
visualization can be performed using any number of libraries such as
matplotlib and vtk.

In addition, I will need a rather large repository of music files. Luckily,
I have plenty myself as well as access to Spotify Premium.

\begin{thebibliography}{9}
\bibitem{jeong} Jeong, Dasaem, and Juhan Nam. "Visualizing music in its entirety
using acoustic features: Music flowgram." in Proceedings of the International
Conference on Technologies for Music Notation and Representation-TENOR2016,
Anglia Ruskin University. Anglia Ruskin University. 2016.
\bibitem{cuadra} De La Cuadra, Patricio, Aaron S. Master, and Craig Sapp,
"Efficient Pitch Detection Techniques for Interactive Music." ICMC. 2001.
\bibitem{rabiner} Rabiner, Lawrence, et al. "A comparative performance study of several
pitch detection algorithms." IEEE Transactions on Acoustics, Speech, and Signal
Processing 24.5 (1976): 399-418.
\bibitem{zhu} Zhu, Yongwei, Mohan S. Kankanhalli, and Sheng Gao. "Music key
detection for musical audio." Multimedia Modelling Conference, 2005. MMM 2005.
Proceedings of the 11th International. IEEE, 2005.
\bibitem{chai} Chai, Wei, and Barry Vercoe. "Detection of Key Change in
Classical Piano Music." ISMIR. 2005.
\bibitem{bello} Bello, Juan Pablo, et al. "A tutorial on onset detection in
music signals." IEEE Transactions on speech and audio processing 13.5 (2005):
1035-1047.
\end{thebibliography}


\end{document}
