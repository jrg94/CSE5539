\documentclass[12pt, a4paper]{article}
\setlength{\oddsidemargin}{0.5cm}
\setlength{\evensidemargin}{0.5cm}
\setlength{\topmargin}{-1.6cm}
\setlength{\leftmargin}{0.5cm}
\setlength{\rightmargin}{0.5cm}
\setlength{\textheight}{24.00cm}
\setlength{\textwidth}{15.00cm}
\parindent 0pt
\parskip 5pt
\pagestyle{plain}

\title{Project Proposal}
\author{}
\date{}

\newcommand{\namelistlabel}[1]{\mbox{#1}\hfil}
\newenvironment{namelist}[1]{%1
\begin{list}{}
    {
        \let\makelabel\namelistlabel
        \settowidth{\labelwidth}{#1}
        \setlength{\leftmargin}{1.1\labelwidth}
    }
  }{%1
\end{list}}

\begin{document}
\maketitle

\begin{namelist}{xxxxxxxxxxxx}
\item[{\bf Title:}]
	Music Data Mining for Data Visualization
\item[{\bf Author:}]
	Jeremy Grifski
\item[{\bf Instructor:}]
	Professor DeLiang Wang
\end{namelist}

\section*{Background}

As a first-year PhD student, my research background is very limited. That said,
I am interested in doing research in the area between music, education, gaming,
and data visualization.

To supplement these interests, I took three game development courses, a
computer graphics course, and a modeling and simulation course during undergrad.
In addition, I recently took a real-time rendering course, and I am
currently taking a data visualization course and a graphics seminar.

In terms of technical experience, I spent two years in industry working for
General Electric Transportation as a part of the Edison Engineering Development
Program. In the span of two years, I managed to rotate through various roles
including software engineer and prognostics engineer. The first role allowed
me to work on camera systems while the second role gave me experience with
some basic data analytics.

\section*{Aim}

The aim of this project is to explore different audio signal processing methods
as a mode of data mining for the purposes of visualization. For example, I am
interested in collecting data such as loudness, onset density, and auditory
roughness~\cite{knuth}. In addition, I'd like to explore pitch and onset
detection.

\section*{Method}

Over the course of the semester, there are several small tasks that I would like
to complete. The list below details a set a features to be implemented in the
final version of the music data mining and visualization tool:

\begin{itemize}
  \item Music directory selection
  \item Recursive music directory traversal
  \item Music file data modeling using Python classes
  \item Music file metadata mapping to data model
  \begin{itemize}
     \item Length
     \item Genre
     \item Artist
     \item Year
     \item Bitrate
   \end{itemize}
\end{itemize}

\section*{Software and Hardware Requirements}

For this project, I intend to use Python for both signal processing and
visualization. Signal processing can be done using the scipy library, and
visualization can be performed using any number of libraries such as
matplotlib and vtk.

In addition, I will need a rather large repository of music files. Luckily,
I have plenty myself as well as access to Spotify Premium.

\begin{thebibliography}{9}
\bibitem{knuth} D. E. Knuth. {\em The \TeX~book.}\/ Addison-Wesley,
Reading, Massachusetts, 1984.
\bibitem{lamport} L. Lamport. {\em \LaTeX~: A Document Preparation
System}.\/ Addison-Wesley, Reading, Massachusetts, 1986.
\bibitem{ken} Ken Wessen, Preparing a thesis using \LaTeX~, private
communication, 1994.
\bibitem{lamport2} L. Lamport. Document Production: Visual
or Logical, {\em Notices of the Amer. Maths. Soc.},\/ Vol. 34,
1987, pp. 621-624.
\end{thebibliography}


\end{document}
