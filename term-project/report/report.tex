\documentclass[journal]{vgtc}                % final (journal style)
%\documentclass[review,journal]{vgtc}         % review (journal style)
%\documentclass[widereview]{vgtc}             % wide-spaced review
%\documentclass[preprint,journal]{vgtc}       % preprint (journal style)

%% Uncomment one of the lines above depending on where your paper is
%% in the conference process. ``review'' and ``widereview'' are for review
%% submission, ``preprint'' is for pre-publication, and the final version
%% doesn't use a specific qualifier.

%% Please use one of the ``review'' options in combination with the
%% assigned online id (see below) ONLY if your paper uses a double blind
%% review process. Some conferences, like IEEE Vis and InfoVis, have NOT
%% in the past.

%% Please note that the use of figures other than the optional teaser is not permitted on the first page
%% of the journal version.  Figures should begin on the second page and be
%% in CMYK or Grey scale format, otherwise, colour shifting may occur
%% during the printing process.  Papers submitted with figures other than the optional teaser on the
%% first page will be refused. Also, the teaser figure should only have the
%% width of the abstract as the template enforces it.

%% These few lines make a distinction between latex and pdflatex calls and they
%% bring in essential packages for graphics and font handling.
%% Note that due to the \DeclareGraphicsExtensions{} call it is no longer necessary
%% to provide the the path and extension of a graphics file:
%% \includegraphics{diamondrule} is completely sufficient.
%%
\ifpdf%                                % if we use pdflatex
  \pdfoutput=1\relax                   % create PDFs from pdfLaTeX
  \pdfcompresslevel=9                  % PDF Compression
  \pdfoptionpdfminorversion=7          % create PDF 1.7
  \ExecuteOptions{pdftex}
  \usepackage{graphicx}                % allow us to embed graphics files
  \DeclareGraphicsExtensions{.pdf,.png,.jpg,.jpeg} % for pdflatex we expect .pdf, .png, or .jpg files
\else%                                 % else we use pure latex
  \ExecuteOptions{dvips}
  \usepackage{graphicx}                % allow us to embed graphics files
  \DeclareGraphicsExtensions{.eps}     % for pure latex we expect eps files
\fi%

%% it is recomended to use ``\autoref{sec:bla}'' instead of ``Fig.~\ref{sec:bla}''
\graphicspath{{figures/}{pictures/}{images/}{../}{./}} % where to search for the images

\usepackage{microtype}                 % use micro-typography (slightly more compact, better to read)
\PassOptionsToPackage{warn}{textcomp}  % to address font issues with \textrightarrow
\usepackage{textcomp}                  % use better special symbols
\usepackage{mathptmx}                  % use matching math font
\usepackage{times}                     % we use Times as the main font
\usepackage{cite}                      % needed to automatically sort the references
\usepackage{tabu}                      % only used for the table example
\usepackage{booktabs}                  % only used for the table example
\usepackage{minted}
%% We encourage the use of mathptmx for consistent usage of times font
%% throughout the proceedings. However, if you encounter conflicts
%% with other math-related packages, you may want to disable it.

%
% Minted Settings
%

\setminted{autogobble}

%% In preprint mode you may define your own headline.
%\preprinttext{To appear in IEEE Transactions on Visualization and Computer Graphics.}

%% If you are submitting a paper to a conference for review with a double
%% blind reviewing process, please replace the value ``0'' below with your
%% OnlineID. Otherwise, you may safely leave it at ``0''.
\onlineid{0}

%% declare the category of your paper, only shown in review mode
\vgtccategory{Research}
%% please declare the paper type of your paper to help reviewers, only shown in review mode
%% choices:
%% * algorithm/technique
%% * application/design study
%% * evaluation
%% * system
%% * theory/model
\vgtcpapertype{application}

%% Paper title.
\title{Genre Comparison through Music Signal Processing}

%% This is how authors are specified in the journal style

%% indicate IEEE Member or Student Member in form indicated below
\author{Jeremy Grifski}
\authorfooter{
%% insert punctuation at end of each item
\item
 Jeremy Grifski is an OSU PhD Student. E-mail: grifski.1@osu.edu.
}

%other entries to be set up for journal
\shortauthortitle{Grifski \MakeLowercase{\textit{et al.}}: Genre Comparison through Music Signal Processing}
%\shortauthortitle{Firstauthor \MakeLowercase{\textit{et al.}}: Paper Title}

%% Abstract section.
\abstract{Duis autem vel eum iriure dolor in hendrerit in vulputate
velit esse molestie consequat, vel illum dolore eu feugiat nulla
facilisis at vero eros et accumsan et iusto odio dignissim qui blandit
praesent luptatum zzril delenit augue duis dolore te feugait nulla
facilisi. Lorem ipsum dolor sit amet, consectetuer adipiscing elit,
sed diam nonummy nibh euismod tincidunt ut laoreet dolore magna
aliquam erat volutpat. Ut wisi enim ad minim veniam, quis nostrud exerci tation ullamcorper
suscipit lobortis nisl ut aliquip ex ea commodo consequat. Duis autem
vel eum iriure dolor in hendrerit in vulputate velit esse molestie
consequat, vel illum dolore eu feugiat nulla facilisis at vero eros et
accumsan et iusto odio dignissim qui blandit praesent luptatum zzril
delenit augue duis dolore te feugait nulla facilisi.%
} % end of abstract

%% Keywords that describe your work. Will show as 'Index Terms' in journal
%% please capitalize first letter and insert punctuation after last keyword
\keywords{Signal Processing, Python, JavaScript, D3}

%% ACM Computing Classification System (CCS).
%% See <http://www.acm.org/class/1998/> for details.
%% The ``\CCScat'' command takes four arguments.

\CCScatlist{ % not used in journal version
 \CCScat{K.6.1}{Management of Computing and Information Systems}%
{Project and People Management}{Life Cycle};
 \CCScat{K.7.m}{The Computing Profession}{Miscellaneous}{Ethics}
}

%% Uncomment below to disable the manuscript note
%\renewcommand{\manuscriptnotetxt}{}

%% Copyright space is enabled by default as required by guidelines.
%% It is disabled by the 'review' option or via the following command:
% \nocopyrightspace

\vgtcinsertpkg

%%%%%%%%%%%%%%%%%%%%%%%%%%%%%%%%%%%%%%%%%%%%%%%%%%%%%%%%%%%%%%%%
%%%%%%%%%%%%%%%%%%%%%% START OF THE PAPER %%%%%%%%%%%%%%%%%%%%%%
%%%%%%%%%%%%%%%%%%%%%%%%%%%%%%%%%%%%%%%%%%%%%%%%%%%%%%%%%%%%%%%%%

\begin{document}

%% The ``\maketitle'' command must be the first command after the
%% ``\begin{document}'' command. It prepares and prints the title block.

%% the only exception to this rule is the \firstsection command
\firstsection{Introduction}

\maketitle

%% \section{Introduction} %for journal use above \firstsection{..} instead
This template is for papers of VGTC-sponsored conferences such as IEEE VIS, IEEE VR, and ISMAR which are published as special issues of TVCG. The template does not contain the respective dates of the conference/journal issue, these will be entered by IEEE as part of the publication production process. Therefore, \textbf{please leave the copyright statement at the bottom-left of this first page untouched}.

\section{Data Collection}

In order to visualize the differences in music genre, I had to first collect
some data. Fortunately, I had quite the collection of songs from iTunes purchases
over the years. In particular, I had 4,194 songs spread across 983 folders.

From there, I had two challenges: choosing which file format to process and
selecting what data to collect from those files. Of the 4,194 songs, roughly
1,800 of them were in the M4A format, so I chose to parse that format.

In terms of data collection, I decided to mine several fields from the M4A files
such as:

\begin{table}[h]
  \caption{Data Fields}
  \label{tab:fields}
  \scriptsize%
	\centering%
  \begin{tabu}{l l}
  \toprule
    field & type \\
  \midrule
  Title & String \\
  Genre & String \\
  Artist & String \\
  Album & String \\
  Purchase Date & Date \\
  Release Date & Date \\
  Track Number & Integer \\
  Sample Rate (Hz) & Integer \\
  Sample Size (bits) & Integer \\
  Duration (HH:MM:SS) & String \\
  Number of Channels & Integer \\
  Average dBFS & Float \\
  Max dBFS & Float \\
  RMS & Integer \\
  Max Amplitude & Integer \\
  \midrule
  \end{tabu}%
\end{table}

In the following sections, I'll discuss how I parsed the M4A files, and how
I aggregated that data for visualization.

\subsection{File Parsing}

The first major challenge in this project was finding a way to collect
information from the M4A file format. Unfortunately, I was unable to find a
tool which could easily glean all the data I wanted from an M4A file. As a result,
I decided to write my own file parsing code in Python.

With the aid of the Quicktime documentation, I was able to write a Python script
which could roughly interpret the various atoms of an M4A file. In particular,
I wrote enough code to parse all of the following atoms:

\begin{table}[h]
  \caption{All Parsed Atoms}
  \label{tab:atoms}
  \scriptsize%
	\centering%
  \begin{tabu}{l l}
  \toprule
    atom & description \\
  \midrule
    moov & the movie atom \\
    trak & the trak atom \\
    mdia & the media atom \\
    minf & the media information atom \\
    stbl & the sample table atom \\
    udta & the user data atom \\
    dinf & the data information atom \\
    pinf & - \\
    schi & - \\
    ftyp & the file type atom \\
    mvhd & the movie header atom \\
    hdlr & the handler atom \\
    mdhd & the media header atom \\
    dref & the data reference atom \\
    smhd & the sound media information header atom \\
    stco & the chunk offset table atom \\
    stsc & the sample-to-chunk table atom \\
    stsd & the standard description atom \\
    esds & the elementary stream descriptor atom \\
    stsz & the sample size table atom \\
    stts & the time-to-sample table atom \\
    tkhd & the track header atom \\
    meta & the meta data atom \\
    ilst & the item list atom \\
    cpil & the compilation atom \\
    gnre & the content genre (i.e. rock) atom \\
    rtng & the content rating (i.e. explicit) atom \\
    stik & the media type (i.e. movie) atom \\
    mdat & the media data atom \\
    frma & the data format atom \\
  \midrule
  \end{tabu}%
\end{table}

An atom is a section of an M4A file that stores information based on some accepted
standard. Some atoms contain atoms, so parsing can be challenging. Fortunately,
Python makes it relatively easy to read and parse byte streams. The challenge
was properly mapping each atom based on the standard.

In total, the M4A parsing code is roughly 800 lines of code, yet it is still
incomplete. In particular, I was unable to parse the actual audio data as
M4A files are in many cases compressed. To account for this setback, I leveraged
a Python library called pydub which uses FFmpeg to parse audio files of many
formats including M4A.

\subsection{Data Aggregation}

Being able to parse an M4A file and being able to use that data for visualization
are two separate challenges. As a result, I had to find a way to aggregate the
parsed data in an easy-to-read format. With the target environment being D3
and JavaScript, I chose to format all the data as JSON files. JSON is easy
to load in JavaScript and even easier to change as needed. An example JSON
might look something like the following:

\begin{minted}{json}
{
    "path": "E:\\Plex\\Music
    \\A Loss for Words\\The Kids Can't Lose
    \\01 Stamp of Approval.m4a",
    "genre": "Alternative",
    "title": "Stamp of Approval",
    "artist": "A Loss for Words",
    "album": "The Kids Can't Lose",
    "track_number": 1,
    "total_tracks": 11,
    "content_rating": null,
    "owner": "Jeremy Grifski",
    "release_date": "2009-05-12T07:00:00Z",
    "purchase_date": "2014-02-10 16:44:15",
    "sample_rate": 44100,
    "sample_size": 16,
    "length": "00:03:06",
    "number_of_channels": 2,
    "dBFS": -11.080573225629289,
    "max_dBFS": 0.0,
    "rms": 599654648,
    "max_amplitude": 2147483648,
    "_raw_json": null,
    "_chunk_offset_table": null,
    "_sample_to_chunk_table": null,
    "_sample_size_table": null,
    "_time_to_sample_table": null,
    "_sample_description_table": null,
    "_music_data": null
}
\end{minted}

With a JSON like this, it's very easy to retrieve information about a song like
its genre or title. In addition, the JSON can be easily expanded to include
information like the occurences of certain pitches.

In order to generate the JSON, I created two Python classes: MusicFile and
MusicFileSet. A MusicFile provides music file data storage as seen in the JSON
above. In addition, the MusicFile class provides extended functionality like
the ability to convert an M4A to a WAV file and the ability to persist the
parsed atom data to JSON.

Each MusicFile can then be stored and manipulated in an instance of the
MusicFileSet class. The MusicFileSet class is essentially a list wrapper, but
it provides several key functionalities such as:

\begin{itemize}
  \item the ability to load all M4A files from a directory recursively
  \item the ability to filter a MusicFileSet by genre or size
  \item the ability to dump the entire set to JSON
\end{itemize}

Naturally, I leveraged the MusicFileSet class to build a large collection of
data files which encompass various sample sizes and genres. In total, I was
able to generate 33 data sets.

\section{Data Visualization}

The goal of this project was to learn some signal processing while trying to
find interesting trends in genres of music. To do so, I felt it was appropriate
to try to plot some of the data I collected in D3.

To begin, I felt it was important to get an idea of what kind of data I was
using. Since I was planning on working with genre data, I decided to create
a histogram of the various genres. Needless to say, the results were skewed
toward my favorite genres:

\begin{figure}[h]
 \centering % avoid the use of \begin{center}...\end{center} and use \centering instead (more compact)
 \includegraphics[width=\columnwidth]{genre-histogram}
 \caption{A genre histogram from \cite{Grifski:2019} of all the files in the dataset.}
 \label{fig:genres}
\end{figure}

The visualization above was rendered from the following table of data:

\begin{table}[h]
  \caption{Genre Totals}
  \label{tab:genres}
  \scriptsize%
	\centering%
  \begin{tabu}{l c}
  \toprule
    genre & count \\
  \midrule
  Alternative & 448 \\
  Rock & 368 \\
  None & 353 \\
  Pop & 207 \\
  Jazz & 130 \\
  Soundtrack & 107 \\
  Country & 88 \\
  Comedy & 44 \\
  Metal & 42 \\
  Punk & 31 \\
  Rap & 20 \\
  Classical & 12 \\
  Electronic & 7 \\
  Vocal & 6 \\
  Dance & 3 \\
  Hip-Hop & 2 \\
  Instrumental & 1 \\
  \midrule
  \end{tabu}%
\end{table}

\section{Bibliography Instructions}

\begin{itemize}
\item Sort all bibliographic entries alphabetically but the last name of the first author. This \LaTeX/bib\TeX\ template takes care of this sorting automatically.
\item Merge multiple references into one; e.\,g., use \cite{Max:1995:OMF,Kitware:2003} (not \cite{Kitware:2003}\cite{Max:1995:OMF}). Within each set of multiple references, the references should be sorted in ascending order. This \LaTeX/bib\TeX\ template takes care of both the merging and the sorting automatically.
\item Verify all data obtained from digital libraries, even ACM's DL and IEEE Xplore  etc.\ are sometimes wrong or incomplete.
\item Do not trust bibliographic data from other services such as Mendeley.com, Google Scholar, or similar; these are even more likely to be incorrect or incomplete.
\item Articles in journal---items to include:
  \begin{itemize}
  \item author names
	\item title
	\item journal name
	\item year
	\item volume
	\item number
	\item month of publication as variable name (i.\,e., \{jan\} for January, etc.; month ranges using \{jan \#\{/\}\# feb\} or \{jan \#\{-{}-\}\# feb\})
  \end{itemize}
\item use journal names in proper style: correct: ``IEEE Transactions on Visualization and Computer Graphics'', incorrect: ``Visualization and Computer Graphics, IEEE Transactions on''
\item Papers in proceedings---items to include:
  \begin{itemize}
  \item author names
	\item title
	\item abbreviated proceedings name: e.\,g., ``Proc.\textbackslash{} CONF\_ACRONYNM'' without the year; example: ``Proc.\textbackslash{} CHI'', ``Proc.\textbackslash{} 3DUI'', ``Proc.\textbackslash{} Eurographics'', ``Proc.\textbackslash{} EuroVis''
	\item year
	\item publisher
	\item town with country of publisher (the town can be abbreviated for well-known towns such as New York or Berlin)
  \end{itemize}
\item article/paper title convention: refrain from using curly brackets, except for acronyms/proper names/words following dashes/question marks etc.; example:
\begin{itemize}
	\item paper ``Marching Cubes: A High Resolution 3D Surface Construction Algorithm''
	\item should be entered as ``\{M\}arching \{C\}ubes: A High Resolution \{3D\} Surface Construction Algorithm'' or  ``\{M\}arching \{C\}ubes: A high resolution \{3D\} surface construction algorithm''
	\item will be typeset as ``Marching Cubes: A high resolution 3D surface construction algorithm''
\end{itemize}
\item for all entries
\begin{itemize}
	\item DOI can be entered in the DOI field as plain DOI number or as DOI url; alternative: a url in the URL field
	\item provide full page ranges AA-{}-BB
\end{itemize}
\item when citing references, do not use the reference as a sentence object; e.\,g., wrong: ``In \cite{Lorensen:1987:MCA} the authors describe \dots'', correct: ``Lorensen and Cline \cite{Lorensen:1987:MCA} describe \dots''
\end{itemize}

\section{Conclusion}

Lorem ipsum dolor sit amet, consetetur sadipscing elitr, sed diam
nonumy eirmod tempor invidunt ut labore et dolore magna aliquyam erat,
sed diam voluptua. At vero eos et accusam et justo duo dolores et ea
rebum. Stet clita kasd gubergren, no sea takimata sanctus est Lorem
ipsum dolor sit amet. Lorem ipsum dolor sit amet, consetetur
sadipscing elitr, sed diam nonumy eirmod tempor invidunt ut labore et
dolore magna aliquyam erat, sed diam voluptua. At vero eos et accusam
et justo duo dolores et ea rebum. Stet clita kasd gubergren, no sea
takimata sanctus est Lorem ipsum dolor sit amet. Lorem ipsum dolor sit
amet, consetetur sadipscing elitr, sed diam nonumy eirmod tempor
invidunt ut labore et dolore magna aliquyam erat, sed diam
voluptua. At vero eos et accusam et justo duo dolores et ea
rebum.


%% if specified like this the section will be committed in review mode
\acknowledgments{
The authors wish to thank A, B, and C. This work was supported in part by
a grant from XYZ (\# 12345-67890).}

%\bibliographystyle{abbrv}
\bibliographystyle{abbrv-doi}
%\bibliographystyle{abbrv-doi-narrow}
%\bibliographystyle{abbrv-doi-hyperref}
%\bibliographystyle{abbrv-doi-hyperref-narrow}

\bibliography{report}
\end{document}
